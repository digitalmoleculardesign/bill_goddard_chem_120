\section{References}
\begin{enumerate}
\item  H. Hellmann, Z. Physik, Vol. 85, page 180 (1933).
      
\item  K. Ruedenberg, Rev. Mod. Phys., Vol. 34, page 326 (1962).
      
\item  C. W. Wilson, Jr., and W. A. Goddard III, Chem. Phys. Lett., 
Vol. 5, page 45 (1970).
    
\item  C. W. Wilson, Jr., and W. A. Goddard III, Theor. Chim. Acta, 
Vol. 26, pages 195 and 211 (1972).
    
\item  M. J. Feinberg and K. Ruedenberg, J. Chem. Phys., Vol. 54, 
page 1495 (1971).
    
\item  M. J. Feinberg, K. Ruedenberg, and E. L. Mehler, Advan. Quant. 
Chem., Vol. 5, page 28 (1970).
    
\item  R. F. Bader and A. D. Baudraut, J. Chem. Phys., Vol. 49, 
page 1653 (1968).
    
\item  J. C. Slater, in Quantum Theory of Molecules and Solids, 
Vol. I, page 266.
\end{enumerate}

\section{Exercise}
    
\begin{enumerate}
\item Experimentally, and theoretically, the molecules Li$^+_2$ 
Na$^+_2$, and K$^+_2$, with one-electron bonds, are found to have stronger 
bonds than the corresponding molecules L$_2$, Na$_2$, and K$_2$ having 
two-electron bonds.  Explain the origin of this effect.
\end{enumerate}

\begin{enumerate}
\item From Appendix D. Evaluate
\begin{equation}
\langle \chi_\ell(1) \chi_\ell(1) \vert {1 \over r_{12}} \vert 
\chi_r(2) \chi_r(2) \rangle
\end{equation}

\item From Appendix E. Evaluate $\rho^x_g$ and $\rho^{cl}$ 
at the bond midpoint for $R = 2.5a_0$. The answer should be
\begin{equation}
\rho^x_g \left( {R \over 2} \right) = {0.033 \over \pi}
\end{equation}
and
\begin{equation}
\rho^{cl} \left( {R \over 2} \right) = {0.286 \over \pi}
\end{equation}

\item Evaluate $\rho^x_g$ and $\rho^{cl}$ at the left nucleus. The 
answer should be
\begin{equation}
\rho^x_g (0) = {-0.10 \over \pi}
\end{equation}
and 
\begin{equation}
\rho^{cl} (0) = {0.50 \over \pi}.
\end{equation}
\end{enumerate}

