\bigskip

\noindent{{\bf 9.5  References}}

\item {1.}  L. Pauling, in Nature of the Chemical Bond, Cornell 
University Press, Ithaca, New York, 1960, Third Edition.

\item {2.} Geometry and bond energy from JANAF, 1964.

\item {3.} NBS Technical Note 270-3.

\item {4.} R. Hultgren, P. D. Desai, D. T. Hawkins, M. Gleiser, K. K. Kelley,
and D. D. Wagman, in Selected Values of the Thermodynamic Properties
of the Elements, American Society for Metals, Metals Park, Ohio, 1973.

\item {5.} W. P. Pearson, in Handbook of Lattice Spacings and Structure 
of Metals, Pergamon Press, New York, 1968.

\item {6.} (a) J. Donohue, in The Structure of the Elements, Wiley, New 
York, 1974.  (b) A. F. Wells, in Structural and Inorganic Chemistry, 
Pergamon Press, New York, 1975.

\item {7.} I. Barin and O. Knacke, in Thermochemical Properties of Inorganic
Substances, Springer, 1973.

\item {8.} JANAF.

\item {9.} (a) LB II/7.  (b) JPCRD.

\item {10.} Klemperer, et al.

\bigskip

\noindent{{\bf 9.6  Exercises}}

\item {1.} The original Pauling scale of electronegativities was 
developed when there were little experimental data. (a) Using (6) and Table 
9-3, rederive the electronegativities.  (b) Using (9) and Table 9-6, 
rederive the electronegativities.  (c) The results of (a) and (b) may not
agree. If not, consider re-defining either (8) or (9), or both, so as to 
get consistent results.  (d) What suggestions can you make about the 
relation (11) and electronegativity.

\item {2.} A number of compounds have composition XYO$_3$, where Y has 
a formal charge of +4.  Assuming that Y is octahedral, consider the 
various possible coordinations of X, 1 through 12, and determine which 
leads to electrostatic balance, assume all O are equivalent. The
structure of perovskite, CaTiO$_3$, which is shared by a number of 
minerals has six cations coordinated to each oxygen,  including
SrRiO$_3$, SrFeO$_3$, BaTiO3, KTaO$_3$, NaTaO$_3$, KNb0$_3$, PbTiO$_3$, 
and many others.  Given this information, try to construct the 
structure, without looking it up.  Use BaTiO$_3$ as the prototype. This is an 
interesting case because, below 380$^{\circ}$K it transforms to a lower 
symmetry involving a concerted motion of the Ti in the $z$
direction.  This leads to a net polarization, or dipole moment, in 
the $z$ direction and hence a macroscopic electric field.  Such materials 
are called ferroelectrics and are useful as nonlinear optical materials.

\item { } Answer: Since Ti is octahedral, we have S$_{Ti} = 2/3$.  Consider 
the various coordinations of Ba, calculate S$_{Ba}$ and then determine the 
number of Ti, N$_{Ti}$, and Ba, N$_{Ba}$ about each O. Since we assume that 
every O is equivalent, then both N$_{Ti}$ and N$_{Ba}$ must be
nonzero, leading to the possibilities in Table 9-21.

\item {3.}	A number of compounds have composition X$_1$Y$_2$O$_4$, where Y 
is nominally Y$^{3+}$.  Assuming that Y is octahedral, consider the various 
coordinations for X = 4, 6, 8, 12, and for each case determine how many X 
and Y must be coordinated to the O for electrostatic balance.  Assume all
O are equivalent. If more than one choice is possible, choose the one 
with the fewest metals around each O.  The spinel structure, e.g., 
MgAl$_2$O$_4$ has four cations coordinated to each O.  What is the 
coordination of each Mg?  See if you can construct the spinel
structure without looking it up.
\vfill\eject

\baselineskip=14pt
