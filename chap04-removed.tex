\section{References}
\begin{enumerate}
\item J. L. Slater, Phys. Rev. {\bf 34}, 1293 (1920).
\end{enumerate}

\section{Exercises}
\begin{enumerate}
\item From Section 4.4. Calculate $S^-\chi_{11}$, $S^-\chi_{10}$,
$S^-\chi_{11}$, and $S^-\chi_{00}$ for the wavefunctions in
(\ref{chap4-eqno24}) and (\ref{chap4-eqno25}). Use the results to show
that the wavefunctions are angular momentum eigenstates.

\item Show that the wavefunction (\ref{chap4-eqno33}) is a spin
eigenfunction with $S = 1/2(n - m)$.
\end{enumerate}


\subsubsection{Valence Bond Spin Eigenfunctions}

Consider the case in which we want to obtain the spin eigenfunctions 
for a singlet state for some large systems, with an even number of electrons.  
It is easy to get one function, namely the $Y1$ function
\begin{equation}
\left( \alpha \beta - \beta \alpha \right) \left( \alpha \beta - 
\beta \alpha \right) \left( \alpha \beta - \beta \alpha \right) 
\cdots
\label{chap4app-eqno22}
\end{equation}
Applying ${\hat S}^+$ to this function yields zero, and hence, it is
clearly a singlet function, $S = 0$.  However, for $N > 2$, there are
other linearly independent spin functions with $S = 0$, as indicated
in Figure \ref{fig4-02}. These can be obtained as outlined
earlier. However, these procedures of building up spin functions get
very time-consuming for large $N$.

An alternate approach is to use functions like (\ref{chap4app-eqno22}) but with the 
electrons renumbered.  For example, with $N = 4$ we get three such functions
\begin{eqnarray}
\chi_{12,34} = \left[ \alpha ( 1 ) \beta ( 2 ) - \beta (1) \alpha (2) 
\right] \left[ \alpha (3) \beta (4) - \beta (3) \alpha (4) \right] =
& 1 --- 2\cr
& 4 --- 3\cr
%
\label{chap4app-eqno23}
\end{eqnarray}
\begin{eqnarray}
\chi_{13,24} = \left( \alpha (1) \beta (3) - \beta (1) \alpha (3) 
\right] \left[ \alpha (2) \beta (4) - \beta (2) \alpha (4) \right] =
& 1 {~~~~~} 2\cr
& 4 {~~~~~} 3\cr
%
\label{chap4app-eqno24}
\end{eqnarray}
and
\begin{eqnarray}
\chi_{14,23} = \left[ \alpha (1) \beta (4) - \beta (1) \alpha (4) 
\right] \left[ \alpha (2) \beta (3) - \beta (2) \alpha (3) \right] = 
& 1 {~~~~~} 2\cr
& 4 {~~~~~} 3\cr
%
\label{chap4app-eqno25}
\end{eqnarray}
These functions are denoted by diagrams in which the electrons are
spread in sequence around the periphery of a circle and lines are
drawn connecting singlet paired electrons, as shown in
(\ref{chap4app-eqno23}), (\ref{chap4app-eqno24}), and
(\ref{chap4app-eqno25}).  Unfortunately, the functions in this set are
not linearly independent.  For example,
\begin{equation}
\chi_{12,34} + \chi_{14,23} = \chi_{13,24} .
\end{equation}
However, Rumer showed that if diagrams with crossed line, as in
(\ref{chap4app-eqno24}), are eliminated, then the remaining spin
eigenfunctions are linearly independent.  For example, the valence
bond (VB), spin eigenfunctions for a six-electron single are

%% RPM something goes here

That is, there are five linearly independent spin functions, just as 
indicated in Figure \ref{fig4-02}

These spin eigenfunctions are not orthogonal. For example,
\begin{equation}
\langle \chi_{12,34} | \chi_{14,23} \rangle = - 1
\end{equation}
note that these functions are not normalized.  However, they are easy to 
use in buiIding up the type of many-electron wavefunctions referred to as 
valence bond wavefunctions.  For this reason, the spin eigenfunction of type 
(\ref{chap4app-eqno22}) are referred to as valence bond spin eigenfunctions.

For cases with $S \not= 0$, the valence bond spin eigenfunction are 
obtained by starting with the single valence bond spin functions for 
an $N + 2S$ electron system and deleting the last $2S$ electrons, and then, 
retaining the cases having $2S$ unpaired orbitals.  For example, the valence 
bond spin functions for $N=3$ and $S= 1/2$ are
\begin{equation}
\chi_{12,3} = {1 \atop ~~} ~~~~~ {2 \atop {\cdot 3}} = \left( 
\alpha \beta - \beta \alpha \right) \alpha
\end{equation}
and
\begin{equation}
\chi_{23,1} = {{1\cdot} \atop ~~} {~~~~~} {2 \atop 3} = 
\alpha \left( \alpha \beta - \beta \alpha \right)
\end{equation}
obtained from (\ref{chap4app-eqno23}) and (\ref{chap4app-eqno25}),
respectively.  As another example, to obtain the valence bond spin
functions for a four-electron triplet, we start with the $S = 0$
functions for $N = 6$ of (\ref{chap4app-eqno26}), and delete electrons
five and six.  This leads to three cases with two unpaired electrons
\begin{equation}
\chi_{23,1,4} = {{\d 1} \atop {\dot 4}} {~~~~~} {2 \atop 3} = \alpha 
\alpha \beta \alpha - \alpha \beta \alpha \alpha
\end{equation}
\begin{equation}
\chi_{12,3,4} = {1 \atop {\dot 4}} {~~~~~} {2 \atop {\dot 3}} = \left( 
\alpha \beta - \beta \alpha \right) \alpha \alpha
\end{equation}
\begin{equation}
\chi_{14,2,3} = {1 \atop 4} {~~~~~} {{\d 2} \atop {\dot 3}} = \alpha \alpha 
\alpha \beta - \beta \alpha \alpha \alpha .
\end{equation}

\section{References}
\begin{enumerate}
\item A. Young, in Proc. London Math. Soc., vol. 34, page 196 
(1931).  A particularly lucid description of Young's work is given by 
D. E. Rutherford, in Substitutional Analysis, Eidenburgh University 
Press (1948).

\item T. Yamanuchi, in Proc. Phys. Math. Soc. Japan, vol. 18, 
page 623 (1936).

\item See also M. Kotani, A. Amemiya, E. Ishiguro, and T. Kimur, 
in Table of Molecular Intervals, Mazurzen Co., Ltd., Tokyo, Chapter 
1 (1955).

\item From F. W. Bobrowicz, Ph.D. Thesis, California Institute 
of Technology, pages 93 through 101 (1974).
\end{enumerate}
