
\noindent{{\bf 14.7  References}}

\item {1.} Mark McAdon and William A. Goddard, to be published.

\item {2.} R. Hultgren, P. D Desai, D. T. Hawkins, M. Gleiser, K. K. 
Kelley, and D. D. Wagman, in Selected Values of the Thermodynamic 
Properties of the Elements, American Society for Metal, Metal Parks, 
Ohio, 1973.

\item {3.} W. P. Pearson, in Handbook of Lattice Spacings and 
Structure of Metals, Pergamon Press, New York, 1968.

\item {4.} (a) J. Donohue, in The Structure of the Elements, Wiley, 
New York, 1974.  (b) A. F. Wells, in Structural and Inorganic 
Chemistry, Pergamon Press, New York, 1975.

\item {5.} I. Barin and O. Knacke, in Thermochemical Properties of 
Inorganic Substances, Springer, 1973.

\item {6.} JANAF.

\item {7.} (a) LB II/7.  (b) JPCRD.

\item {8.} NBS Technical Note 270-3.

\bigskip

\noindent{{\bf 14.8  Exercises}

\item{1.} Consider body-centered cubic, hexagonal closest-packed, and 
face-centered cubic structures.  Show where in the unit cell electrons 
would be if there are one-electron bonds with the electrons localized 
in tetrahedra.
	
\item{2.} Same as problem 1, but with the electrons in octahedra.

\item{3.} Assuming the one-electron bonding picture presented for alkalis 
is correct, extend this picture to systems such as Be or Mg.  Analyze which 
crystal structure would seem to be favorable.

\item{4.} Analyze the bonding of Al, Ga, In, Tl, and the properties 
of these systems using the one-electron bond 
model described for alkali atoms.

\bigskip
