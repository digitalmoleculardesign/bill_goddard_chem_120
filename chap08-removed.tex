\bigskip

\noindent{{\bf 8.4.  References}}

\item {1.} A. Redondo and W. A. Goddard III, in J. Vac. Sci. TechnoL., 
volume 21, page 344 )(1982).

\bigskip

\noindent{{\bf 8.5  Exercises}}

\item {1.} Consider the (100), (110), and (1111 surfaces of GaAs but indicate
which surface atoms are Ge and which are As. Some surfaces are
polar in that they lead only to Ga or to As atoms on the perfect 
surface. These surfaces tend to not be the most stable.  Based on such
reasoning, which surface is most stable?

\item {2.} Consider the (110) surface of GaAs. Using chemical reasoning, 
predict how the surface atoms will distort from the tetrahedral orientation 
characteristic of the bulk.

\item {3.}  (a) Consider a vacancy in Si, that is, one Si atom is missing. 
Considering the electrons on the atoms adjacent to the vacancy, predict 
how these four electrons should pair up.  Which spin state is best?  How 
would these atoms distort to maximize bonding?  (b) 
Now consider two adjacent vacancies, called a divacancy. Compare the 
energy for the divacancy with that of two separated vacancies.  Which is 
better and by how much?

\item {4.}  (a) Consider a Ga vacancy in GaAs.  What should the bonding 
be like?  What spin state should be optimum? How should the atoms distort?
(b) Same questions for an As vacancy.

\item {5.} Consider a II-VI compound, e.g., ZnSe, with the diamond-type 
structure, i.e., sphalerite. Describe the average character of the 
bonds.  Describe the bonding and spin for Zn and Se vacancies, as in 
problem 4 for GaAs.

\bigskip
