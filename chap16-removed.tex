\bigskip

\noindent{{\bf 16.10  References}}

\item {1.} Cotton, in Chemical Applications of Group Theory, 1963.

\item {2.} Tinkham, in Group Theory and Quantum Mechanics, 1964.

\item {3.} Hamermesh, in Group Theory, 1962.

\item {4.} R. S. Mulliken, in Journal of Chemical Physics, Volume 
23, page 1997, 1995.

\item {5.} P. J. Hay and W. A. Goddard III, in Chemical Physics 
Letters, Volume 14, page 46, 1972.

\item {6.} Herzberg, in Infrared and Raman Spectra, Volume II.

\item {7.} H. A. Jahn and E. Teller, in Procedures of Royal Society, 
Volume A161, page 220, 1937.

\item {8.} H. A. Jahn, in Procedures of Royal Society, Volume A164, 
page 117 (1938).

\item {9.} Griffith, in Theory of Transition Metal Ions, page 169.

\item {10.} Herzberg, in Infrared and Raman Spectra, Volume III, page 
563.

\item {11.} Herzberg, in Infrared and Raman Spectra, Volume III, page 
569.

\item {12.} Griffith, in Theory of Transition Metal Ions, page 388.

\bigskip

\noindent{{\bf 16.11  Exercises}}

\item {1.} From Section 16.1.  Show the identity $e$ is never conjugate 
to another element of a group.

\item {2.} Show that for an Abelian group, no element if conjugate to 
any other element of the group.

\item {3.} Find the weakest condition ensuring that there is only one 
element in each class of a group.

\item {4.} From Section 16.2.  List the elements of {\bf S}$_4$, 
partition them into classes and decompose each of them into 
combinations of elementary transpositions.

\item {5.} Analyze the classes of {\bf T} and {\bf T}$_d$ using the 
elements of {\bf S}$_4$.

\item {6.} From Section 16.3.  Give examples of molecules that have 
the following point groups: {\bf C}$_3$, {\bf S}$_4$, and {\bf 
D}$_5$.  Show the generating symmetry operations graphically.  Hint: 
First think of molecules with higher symmetry, e.g., {\bf C}$_{3v}$ 
for {\bf C}$_3$, {\bf T}$_d$ or {\bf D}$_{2d}$ for {\bf S}$_4$, and 
{\bf D}$_{5h}$ for {\bf D}$_5$, and then find a way to lower the 
symmetry.

\item {7.} Find the classes of {\bf C}$_{5n}$ and {\bf D}$_){5d}$.

\item {8.} Find the classes of {\bf T}$_h$.

\item {9.} From Section 16.4.  Show that for the symmetry group {\bf 
S}$_N$ with $N \leq 2$ there are two, and only two, irreducible 
representations of order 1.

\item {10.}  From Section 16.5.  Determine the splitting pattern for 
$f$ functions, all seven, in {\bf O}$_h$, {\bf T}$_d$, {\bf D}$_{4h}$, 
and {\bf C}$_{2v}$ symmetries.

\item {11.} (a) Consider staggered ethane, {\bf D}$_{3d}$, and analyze 
the symmetries of the occupied orbitals.  (2) Do the same for eclipsed 
ethane, {\bf D}$_{3h}$.  (c) As one rotates between the eclipsed and 
staggered forms of ethane, the symmetry is {\bf D}$_3$.  Find the 
correlations among the occupied orbitals in these three symmetries.

\item {12.}  Consider the benzene molecule.  (a) Find the symmetries 
of the six {\bf C}$_{1s}$ orbitals, the six {\bf CC}$\sigma$ bonding 
orbitals, and the six CH bonding orbitals.  (b) Consider a 
six-dimensional basis formed by a single $p_z$ orbital on each {\bf 
C}.  Reduce this representation.  Indicate the nodal pattern for each 
of the six orbitals.  Predict the three orbitals that are occupied in 
the molecular orbital wavefunction.   Does the total wavefunction 
transform according to the $A_g$ irreducible representation?

\item {13.} Draw vibrational diagrams for CH$_4$ analogous to Figure 
16-40 for BH$_3$.

\item {14.} Draw vibrational diagrams for C$_2$H$_4$ analogous to 
Figure 16-40 for BH$_3$.

\item {15.} Analyze the vibrations of SF$_6$, which has {\bf O}$_h$ 
symmetry.

\item {16.} Construct symmetry functions for the $\pi$ system of 
benzene, allowing one $p_z$ orbital on each carbon.  Hint: Although 
the symmetry group is {\bf D}$_{6h}$, you might first use the 
characters of {\bf C}$_6$ to obtain six eigenfunctions of {\bf 
C}$_6$.  Then consider the effect of $\sigma_v$, i.e., {\bf C}$_{6v}$, 
which combines some of these functions into two dimensional 
representations.  Then consider the effect of adding $\sigma_h$ to 
obtain {\bf D}$_{6h}$.

\item {17.} From Section 16.6.  Show that equation (67) transforms 
according to $A_1$.

\item {18.} Consider the symmetric direct product representation.  
Derive the form of the representation matrix $D^{[\mu \times 
\mu]}(R)$, and the form of the character in equation (68).

\item {19.} From Section 16.8.  Consider four ground state H atoms 
in {\bf T}$_d$ symmetry.  Derive the symmetries of the singlet, 
triplet, and quintet states.  For each state, determine which 
distortions lift the degeneracy.  If more than one distortion would 
do, use simple bonding considerations to predict the optimum 
distortion.  Note, four H atoms in this geometry is not a very 
interesting problem.  However, starting with the crystal structure of 
Si, Ge, or diamond, and removing one atom to form a vacancy leads to 
an exactly analogous system.

\item {20.} From Section 16.9.  If $O\epsilon\{O\{U\}\}$, then there 
exists a $U$ such that $H^{\prime} = UHU^{\dag}=-H$.  Show that there 
can be no such $U \epsilon SU(2)$.

\item {21.} If two H$_2$ molecules approach in a {\bf C}$_{2v}$ 
configuration, the lowest states of the {\bf D}$_{3h}$ configuration 
are $^3A_2^{\prime}$, lowest, $^1E_1$, and $^1A_1$, but only $^1E_1$ 
comes from the ground state of two H$_2$ molecules
\vfill\eject
\hfill
\vskip 2.25truein
\noindent
If we include spin orbit coupling, does the state coming from 
${^1\sum}^+_g + {^1\sum}_g^+$ combine into one of the components of 
${^3A}_2^{\prime}$ of {\bf D}$_{3h}$?

\bigskip

\baselineskip=12pt


\bigskip

\noindent{{\bf 16.13.3  Exercises}}

\item {1.} For this appendix, Appendix 16.13.2.  Consider three-dimensional 
Euclidean space.  Construct the orthogonal matrix representing a 
rotation around the $x$ axis by $\theta$, followed by a rotation about 
the $y$ axis by $\varphi$.

\item {2.} Decompose the inversion operation into a product of a 
rotation and a reflection.  What are the component matrices? 

\bigskip

